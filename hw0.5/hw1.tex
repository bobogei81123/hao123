\documentclass[12pt, a4paper]{article}

\usepackage[hmargin=2.5cm, vmargin=2cm]{geometry}
\usepackage{amsthm, amssymb, mathtools, yhmath, graphicx}
\usepackage{fontspec, type1cm, titlesec, titling, fancyhdr, tabularx}
\usepackage{color, unicode-math, float, hhline}

\usepackage[abbreviations, per-mode=symbol]{siunitx}
\usepackage[CheckSingle, CJKmath]{xeCJK}
\usepackage{CJKulem}
\usepackage{enumitem}
\usepackage{tikz}
\usepackage{csquotes}
\usepackage{circuitikz}
%\setCJKmainfont[BoldFont=cwTex Q Hei]{cwTex Q Ming}
%\setCJKsansfont[BoldFont=cwTex Q Hei]{cwTex Q Ming}
%\setCJKmonofont[BoldFont=cwTex Q Hei]{cwTex Q Ming}
\setCJKmainfont[BoldFont=cwTeX Q Hei]{cwTeX Q Ming}

\def\normalsize{\fontsize{12}{18}\selectfont}
\def\large{\fontsize{14}{21}\selectfont}
\def\Large{\fontsize{16}{24}\selectfont}
\def\LARGE{\fontsize{18}{27}\selectfont}
\def\huge{\fontsize{20}{30}\selectfont}

%\titleformat{\section}{\bf\Large}{\arabic{section}}{24pt}{}
%\titleformat{\subsection}{\large}{\arabic{subsection}.}{12pt}{}
%\titlespacing*{\subsection}{0pt}{0pt}{1.5ex}

\parindent=24pt

\DeclarePairedDelimiter{\abs}{\lvert}{\rvert}
\DeclarePairedDelimiter{\norm}{\lVert}{\rVert}
\DeclarePairedDelimiter{\inpd}{\langle}{\rangle}
\DeclarePairedDelimiter{\ceil}{\lceil}{\rceil}
\DeclarePairedDelimiter{\floor}{\lfloor}{\rfloor}

\DeclareMathOperator*{\argmax}{arg\,max}

\newcommand{\img}{\mathsf{i}}
\newcommand{\ex}{\mathsf{e}}
\newcommand{\dD}{\mathrm{d}}
\newcommand{\dI}{\,\mathrm{d}}
\setlength{\droptitle}{-1.5cm} %title 與上緣的間距
\title{Algorithm HW\#0.5}
\author{B02901178 江誠敏}

\begin{document}
\maketitle
Collaborators: B02901027 茅耀文
\section{Problem 1.}

We claim that the approximation ratio $R = 3$.
Let $S = \sum g_i$, We prove this by examine two cases:
\begin{enumerate}[label=Case \arabic*:]
  \item $g_i \leq S/2, \, \forall g_i$.

    Let $\Delta_i$ be the difference of the sum of the values between child A and child B
    after the $i$-th gift is assigned.
    We claim that $\abs{\Delta_i} \leq S/2, \, \forall i$. Let $k$ be the one which let $\abs{\Delta_k}$
    obtain the maximum. Then after the $k$-th gift is assigned, the one who receive the gift, say child A,
    has the greater total value (or else $\abs{\Delta_{k-1}}$ is bigger). And right before the gift is assigned,
    child A has less total value (so he received the gift). So $\Delta_k, \Delta_{k-1}$ have different sign
    (or possibly $0$), hence $\abs{\Delta_k} = g_k - \abs{\Delta_{k-1}} \leq g_k$. Thus $\abs{\Delta_k} \leq S/2$.

\end{enumerate}

\section{Problem 2.}
\[ n^n > n! > 2^{3n} > 2^n > (\log_2(n))! > 3n^3 + 1 > \sqrt{n} + 3 = 2^{log_4(n)} > n^{0.01} > \log_2 n = \ln n \]
Here $f(n) > g(n)$ means that $g(n) \in o(f(n))$ and $f(n) = g(n)$ means that $f(n) = \Theta(g(n))$.
Notice that in both case $f(n) = \Omega(g(n))$.

\section{Problem 3.}
\begin{enumerate}
  \item We shall assume that $\inf_{1 \leq n \leq 3} T(n) > 0$, or else there exists a function $T(n)$ 
    such that $T(n) \notin \Theta(f(n)), \forall f$.

    We shall prove that $T(n) \in \Theta\left(n (\log n)^2 \right)$ by induction. \\
    Let \[ c_1 = \min\left(\frac{1}{9 (\log 9)^2} \inf_{3 \leq n \leq 9} T(n),\; \frac{1}{2 \log 3} \right), \quad
    c_2 = \max\left(\frac{1}{3(\log 3)^2}\sup_{3 \leq n \leq 9} T(n),\; \frac{1}{\log 3}\right). \] 
    Then $c_1 > 0$, since $T(n) = 3T(n/3) + n \log n > T(n/3)$, so \\
    $\inf_{3 \leq n \leq 9} T(n) > \inf_{1 \leq n \leq 3} T(n) > 0$.
    Now for $3 \leq n \leq 9$
    \begin{align*}
      T(n) &\geq \frac{n (\log n)^2}{9 (\log 9)^2} T(n) 
      \geq n (\log n)^2 \frac{1}{9 (\log 9)^2} \inf_{3 \leq n \leq 9}T(n)
      \geq c_1 n (\log n)^2 \\
      T(n) &\leq \frac{n (\log n)^2}{3(\log 3)^2} T(n) \leq n (\log n)^2 \left(\frac{1}{3 (\log 3)^2} 
        \sup_{3 \leq x \leq 9} T(n) \right) 
      \leq c_2 n (\log n)^2
    \end{align*}
    So $c_1 n (\log n)^2 \leq T(n) \leq c_2 n (\log n)^2$ for $3 \leq n \leq 9$.

    Assume now $n > 9$ and for all $3 \leq k < n$, $T(k) \leq ck$. Then since $n/3 \geq 3$, we have
    \begin{align*}
      T(n) &= 3 T(n/3) + n \log n \geq c_1 n (\log n/3)^2 + n \log n \\
      &= c_1 n (\log n - \log 3)^2 + n \log n \\
      &= c_1 n (\log n)^2 + \left(  c_1 n (\log 3)^2 + n \log n  - 2c_1 n (\log n) (\log 3)\right) \\
      &\geq c_1 n (\log n)^2 + \left(  n \log n  - 2c_1 n (\log n) (\log 3)\right) \\
      &\geq c_1 n (\log n)^2 + \left(  n \log n  - \frac{2 \log 3}{2 \log 3} n (\log n) \right) \\
      &= c_1 n (\log n)^2
    \end{align*} and
    \begin{align*}
      T(n) &= 3 T(n/3) + n \log n \leq c_2 n (\log n/3)^2 + n \log n \\
      &= c_2 n (\log n - \log 3)^2 + n \log n \\
      &= c_2 n (\log n)^2 - \left( 2c_2 n (\log n) (\log 3) - c_2 n (\log 3)^2 - n \log n \right) \\
      &= c_2 n (\log n)^2 - \left(c_2 n (\log n) (\log 3) - c_2 n (\log 3)^2 \right) - 
      \left(c_2 (\log 3) n (\log n) - n \log n \right) \\
      & \leq c_2 n (\log n)^2
    \end{align*}
    because $\log n \geq \log 3$ and $c_2 \geq 1 / \log 3 \implies c_2 \log 3 \geq  1$.

    So $c_1 n (\log n)^2 \leq T(n) \leq c_2 n (\log n)^2$. 
    Hence by induction, we prove that \\ $T(n) \in \Theta(n (\log n)^2)$.

  \item We shall prove that $T(n) \in \Theta(n^3)$ by induction. \\
    Assume for all $1 \leq k < n$, $T(k) \leq ck^3$. Then
    \begin{align*}
      T(n) &= 4 T(n/2) + n^3 \\
      &\leq 4 c (n/2)^3 + n^3 \\
      &= c n^3/2 + n^3 \\
      &= n^3 + n^3 \\
      &= 2 n^3 = c n^3
    \end{align*}
    Hence by induction, we prove that $T(n) \in O(n^3)$. \\
    Notice also that $T(n) = 4T(n/2) + n^3 \geq n^3$, 
    so $T(n) \in \Omega(n^3)$, hence $T(n) \in \Theta(n^3)$.
\end{enumerate}
\section{Problem 4}
\begin{enumerate}
  \item True. For all $c$, let $c' = \frac{1}{2c}$. 
    Since $f(n) \in o(n^2)$, exists $N$ such that $n > N \implies f(n) \leq c'n^2$.
    Then $n^2 - cf(n) \geq n^2 - n^2/2 = n^2 / 2$. So for $c_0 = 1/2$ and $N_0 = N$, 
    $n \geq N_0 \implies c_0n^2 \leq n^2 - c f(n)$, hence $n^2 - cf(n) \in o(n^2)$.
  \item False. Let $f(n) = n^2 / 2$ and $c = 2$, then
    \begin{itemize}
      \item $2^{f(n)} \in o(2^{n^2})$, since for any $c' > 0$, if $c' \geq 1$ then $2^{f(n)} \leq 2^{n^2}$
        for all $n \geq 0$, so assume that $0 < c' < 1$, then choose $N = \sqrt{2 \log_2 (1/c)}$, for
        $n \geq N$, 
        \[ 2^{n^2/2} \geq 2^{\log_2(1/c)} \geq \frac{1}{c} \implies 2^{f(n)} = 2^{n^2/2} \leq c 2^{n^2} \]
        hence $2^{f(n)} \in o(2^{n^2})$.
      \item If $c = 2$, then $n^2 - cf(n) = n^2 - 2 n^2/2 = 0$, and obviously $0 \notin \Omega(n^2)$.
    \end{itemize}
    Hence we disprove the statement by giving a counter example.
\end{enumerate}

\section{Problem 5}
\begin{enumerate}
  \item Let $f(n) = n^{2n}$, Then 
    \[ T(n) = 2^n n^n T(n/2) \implies T(n) = \frac{n^{2n}}{(n/2)^n} T(n/2) \implies 
      \frac{T(n)}{f(n)} = \frac{T(n/2)}{f(n/2)} = c\] 
    where $c$ is a constant. Let $c'$ be such constant that $c' = c \frac{T(1)}{f(1)}$,  \\
    so $T(n) = c\frac{T(1)}{f(1)} f(n) = c' f(n)$, hence $T(n) \in \Theta(f(n)) = \Theta(n^{2n})$.
  \item The original recursive formula is equivalent to
    \[ T(n) = \frac{n^2}{(n/2)^2} T(n/2) + \frac{n^2}{2 \log_2 n} \implies 
      \frac{T(n)}{n^2} = \frac{T(n/2)}{(n/2)^2} + \frac{1}{2 \log_2 n} \]
    Let $k = \log_2 n$, so
    \[ \frac{T(2^k)}{4^k} = \frac{T(2^{k-1})}{4^{k-1}} + \frac{1}{2k} \]
    If we let $f(k) = T(2^k) / 4^k$, then $f(k) = f(k-1) + 1 / 2k$, with initial condition $f(0) = 
    T(1) = 1$. (or $C_1 \leq f(k) \leq C_2$ if $1 \leq k < 2$.) Hence 
    \[ f(k) = C \sum_{m = 1}^{k} \frac{1}{2m} = 2C \sum_{m = 1}^{k} \frac{1}{m} \]
    and notice that since $1/x$ monotonic decrease, 
    \[ \log(k+1) - \log 2 \leq \int_{2}^{k+1} \frac{1}{x} \dI x \leq \sum_{m = 2}^{k} \frac{1}{m} 
      \leq \int_1^{k} \frac{1}{x} \dI x = \log(k) \]
    and by the fact that $\log(k+1) \leq \log(2k) = \log(k) + \log 2$, 
    so $\sum\limits_{m=1}^{k} 1/m \in \Theta(\log m)$.

    Hence $f(k) \in \Theta(\log k)$ and $T(2^k) = 4^k f(k) \in \Theta(4^k \log k)$. Thus
    \[ T(n) = 4^{\log_2 n} \log (\log_2 n) = n^2 \log \log_2 n \in \Theta(n^2 \log \log n). \]
\end{enumerate}

\section{Problem 5}
\begin{enumerate}
  \item True, $T(n) = \Theta(f(n)) \iff T(n) = \Theta(n^2)$ since $f(n) = \Theta(n^2)$.

    Again since $f(n) = \Theta(n^2)$, then exists $N_1, c_1, N_2, c_2$, $n \geq N_1 \implies f(n) \geq c_1 n^2$
    and $n \geq N_2 \implies f(n) \leq c_2 n^2$, let $N = \max( N_1, N_2 )$, and let 
    \[ d_1 = \min \left( \frac{1}{(2N)^2} \inf_{N \leq n < 2N} f(n), \; 2c_1 \right), \quad 
      d_2 = \max \left( \frac{1}{N^2} \sup_{N \leq n < 2N} f(n), \; 2c_2 \right) \]
    Then for $N \leq n < 2N, d_1 n^2 \leq f(n) \leq d_2 n^2$.

    Now we shall prove that the inequality also holds for any $n \geq 2N$.
    If for $k = n/2$, $d_1 k^2 \leq f(k) \leq d_2 k^2$ holds. Then 
    \[ f(n) \geq 2 d_1 k^2 + c_1 n^2 \geq \frac{1}{2} d_1 n^2 + \frac{1}{2} d_1 n^2 = d_1 n^2 \]
    and
    \[ f(n) \leq 2 d_2 k^2 + c_2 n^2 \geq \frac{1}{2} d_2 n^2 + \frac{1}{2} d_2 n^2 = d_2 n^2 \]
    So $d_1 n^2 \leq f(n) \leq d_2 n^2$. Hence by induction, $f(n) \in \Theta(n^2)$ and the proof is complete.
  \item False, Let $f(n)$ be the function such that $f(2^t) = 4^{k^2}$ if $(k-1)^2 < t \leq k^2$ for all $k \in \mathbb{N}$.
    Then $f(n) \in \Omega(n^2)$, since $f(2^t) = 4^{k^2} > 4^{t} > (2^t)^2$ and hence $f(n) > n^2$.
    Then for any $k \in \mathbb{N}$, we have
    \begin{align*}
      T\left(2^{k^2}\right) &= 2T\left(2^{k^2}/2\right) +
      f\left(2^{k^2}\right) = 2T\left(2^{k^2-1}\right) + f\left(2^{k^2}\right) \\
      &= 4T\left(2^{k^2-2}\right) + 2f\left(2^{k^2 - 1}\right) + f\left(2^{k^2}\right) \\
      &\vdots \\
      &= 2^k T\left(2^{k^2-k}\right) + \sum_{t = 0}^{k} 2^t f\left(2^{k^2-t}\right)
    \end{align*}
    Notice that $(k-1)^2 \leq (k-1)k = k^2-k \leq k^2$ so
    $f\left(2^{k^2-t}\right) = 4^{k^2}$ when $t = k$. Thus
    \[
      T\left(2^{k^2}\right) 
      = 2^k T\left(2^{k^2-k}\right) + \sum_{t = 0}^{k} 2^t f\left(2^{k^2-t}\right) \\
      \geq 2^k 4^{k^2}    \]
    So $T(n) = 2^{\sqrt{\log n}} n^2$ for all $n = 2^{k^2}$. Obviously $2^{\sqrt{\log n}} n^2 \notin O(n^2)$
    since $2^{\sqrt{\log n}} \to \infty$ as $n \to \infty$. Hence $T(n) \notin O(n^2)$, and we disprove the 
    statement by giving a counter example.
\end{enumerate}

\section{Problem 7}
\begin{enumerate}
  \item We shall proof that $T(n) \in \Theta(n \log n)$ by induction. Choose $c'_1, c'_2$ so that
    $c'_1 n \log n \leq T(n) \leq c'_2 n \log n$ for $n \leq 4$, and let
    \[ c_1 = \min\left(c'_1, \frac{2}{\log 4} \right), \quad c_2 = \max\left(c'_2, \frac{2}{\log 4/3} \right) \]
    So $c_1 n \log n \leq T(n) \leq c_2 n \log n$ for all $n \leq 4$.
    
    Now if $n > 4$, Assume that for $k = n/2$, $c_1 k \log k \leq T(k) \leq c_2 k \log k$. Then
    \begin{align*}
      T(n) &= T(a) + T(b) + 2n  \\
      & \geq c_1 a \log a + c_1 b \log b + 2n = c_1 a \log a + c_1 b \log b + 2(a + b)  \\
      & = c_1 a \left( \log(a/n) + \log n + \frac{2}{c_1} \right) + 
      c_1 b \left( \log(b/n) + \log n + \frac{2}{c_2} \right) \\
      &= c_1 (a + b) \log n + \left(\frac{2}{c_1} - \log(n/a) \right) + \left(\frac{2}{c_1} - \log(n/b) \right) \\
      &\geq c_1 n \log n
    \end{align*}
    since $\log(n/a), \log(n/b) \leq \log 4 \leq 2/c_2$.

    Similarly, 
    \begin{align*}
      T(n) &= T(a) + T(b) + 2n  \\
      & \leq c_2 a \log a + c_2 b \log b + 2n \\
      &= c_2 (a + b) \log n - \left(\log(n/a) - \frac{2}{c_2} \right) - \left(\log(n/b) - \frac{2}{c_2}\right) \\
      &\leq c_2 n \log n
    \end{align*}
    since $\log(n/a), \log(n/b) \leq \log(4/3) \leq 2/c_2$.
    Hence by induction, $c_1 n \log n \leq T(n) \leq c_2 n \log n$ and thus $T(n) \in \Theta(n \log n)$.

  \item We shall proof that $T(n) \in \Theta(n^2)$.

    First we show that $a^2 + b^2 \leq 8 n^2 / 9 $. WLOG, assume $a \geq b$. If $a \leq 2n / 3$, then 
    $a^2 + b^2 \leq 2a^2 \leq 8n^2 / 9$. If $a > 2n / 3$, then $b < n/3$, hence $a^2 + b^2 
    \leq (3n/4)^2 + (1/3)^2 = 97n / 144 \leq 8n^2/9$.

    Now Choose $c'$ so that
    $T(n) \leq c' n^2$ for $n \leq 4$, and let $ c = \max(c', 9 ) $
    
    If $n > 4$, Assume that for $k = n/2$, $T(k) \leq c k^2$. Then
    \begin{align*}
      T(n) &= T(a) + T(b) + n^2  \\
      & \leq c a^2 + c b^2 + n^2 = c (a^2 + b^2) + n^2 \\
      & \leq c\left(8n^2 / 9\right) + n^2 \\
      & = cn^2 \left(\frac{8 + 9/c}{9}\right) \\
      & \leq c n^2
    \end{align*}
    since $c \geq 9$, so $T(n) \in O(n^2)$. Notice that $T(n) = T(a) + T(b) + n^2 \geq n^2$, so
    $T(n) \in \Omega(n^2)$ hence $T(n) \in \Theta(n^2)$.


  
\end{enumerate}

\end{document}

